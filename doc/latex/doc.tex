\documentclass[11pt,twoside,a4paper]{article}

\usepackage[USenglish]{babel}
\usepackage[IL2]{fontenc} 
\usepackage[utf8]{inputenc}
\usepackage{graphicx}
\usepackage{url} 
\usepackage{cite}
\usepackage{listings}
\usepackage{color}
\usepackage{float}
\usepackage[colorlinks = true,
            linkcolor = blue,
            urlcolor  = blue,
            citecolor = blue,
            anchorcolor = blue]{hyperref}

\raggedbottom 

\oddsidemargin=0cm 
\evensidemargin=0cm
\textwidth=16.5cm 
\pagestyle{headings}

\title{Rust and Linked Lists}

\author{Martin Sivák, Lukáš Častven\\[2pt]
	{\small Slovenská technická univerzita v Bratislave}\\
	{\small Fakulta informatiky a informačných technológií}\\
	{\small \texttt{xcastven@stuba.sk}}
	}

\date{\small 1 May 2023}

\begin{document}
\pagestyle{plain}

\maketitle
\tableofcontents

\section{Idea}

Linked lists are data structures heavily used for explaining key computer science
concepts to new students in this field and are very popular in academic environments.
They are trivial to implement and have good properties, at least in theory.

However when you want to implement them in Rust, the triviality is quickly lost,
and as we learned, in the real world, linked lists generally suck (there are few
cases when they can be good, but these are rare).

\section{Requirements}

\subsection*{Learn how to implement Linked List}

Based on feedback from our tutors, we learned that implementing linked lists
in Rust isn't as trivial as in other languages. This is due to a fact that
Rust likes all of the memory to have one clear owner \cite{ycombinatorWritingLinked}.
With this statement we mean safe Rust likes that, of course we can always use
unsafe Rust, but then we don't have the benefits that safe Rust gives us.

There is even a whole book about linked lists implementations in Rust. This
book walks through different kinds of linked lists and shows the reader how
to implement them in Rust, and also shows the reader how hard it can be to
write these lists in safe Rust. \cite{rustunofficialIntroductionLearning}

\subsection*{Implement different kinds}

After we understand why a basic linked list is one of the Rusts worst nightmares,
we will implement different kinds of them, and those are:

\begin{enumerate}
      \item Basic Linked List
      \item Immutable Linked List
      \item Thread safe immutable Linked List
      \item And the king of them all \textbf{Doubly} Linked List
\end{enumerate}

\subsection*{Benchmark them in different scenarios}

After our brains have melted from different intricacies of linked lists in Rust,
we will use Rusts benchmarking library Criterion \cite{docsCriterionRust} to
see how our implementations compare in various scenarios against the mighty
Vec and Rusts standard library's Linked List.

\section{How it went}

\subsection*{Learning how to implement linked list}

After the call with our tutors when we decided that we will implement linked
lists in Rust we still didn't believe that it can be that hard.

But we were horribly wrong. We both started reading the book mentioned before,
and realized that Rusts borrow checker really likes for memory to have one clear
owner.

Our learning went something like this

\begin{itemize}
      \item Hmm, don't put everything on stack, put it in $Box<T>$
      \item Why can't we move the value? Oh the $Box<T>$ owns it...
      \item But wait, what if the element is the end of the list?
      \item Define an enum $Node<T>$ which can be inner element, tail, or nothing.
      \item Null pointer optimization? Oh we understand, so redefine $Node<T>$ to
            be an element or nothing.
      \item Did we just reimplement the $Option<T>$? Yes we did...
      \item Ok, now set the next as the previous head. What is this
            \textbf{cannot move out of borrowed content}?
      \item \textbf{mem::replace}? Sweet Python where are you...
      \item $Option.take()$? That makes sense (Dunning-Kruger Peak of Mount Stupid)
      \item Iterators, yeah we want to iterate over our list.
      \item Lifetime?! How are we supposed to know how long this thing will live?!
      \item Pff, easy, it's just syntactic sugar, why is it needed?
      \item That's why it is needed... \emph{Fall down to Dunning-Kruger Valley of Despair}
\end{itemize}

\subsection*{Implementing different kinds}

After few weeks of reading, testing and finally understanding that there must
be only one owner of the memory, but the owners can have different ways of
how to borrow their memory, we implemented these kinds of linked list and gained
new knowledge:

\begin{enumerate}
      \item Linked List: Box is a heap allocation which owns the data
      \item Immutable Linked List: Rc also owns the value but, it has a primitive
            garbage collection principle of counting how many references are pointing
            at it
      \item Thread safe immutable Linked List: Arc is basically the same thing as
            Rc, but it is atomic, which means it is safe to use in multithreaded environments
      \item Doubly Linked List: RefCell, as we understood it, basically allows you
            to apply borrow checkers rules in runtime, however if they are broken,
            the program panics
\end{enumerate}

\section{Benchmarking}

During the implementation of the linked lists we learned from various sources
that linked lists, in majority of scenarios suck\cite{rustunofficialIntroductionLearning}.
Vecs are faster and are easier to work with.

We have two types of benchmarks, first show us why generally linked lists suck,
and the second one is when they don't suck. To determine if they suck (or don't)
they will be compared to a Vec in same scenario.

\pagebreak
\subsection*{Linked Lists Suck}

In this table we can see how long it took to push 100000 integers into a Vec,\\
std::collections::LinkedList, Linked List, Immutable Linked List, Thread Safe
Linked List and Doubly Linked List.
\begin{table}[!h]
      \begin{center}
            \caption{Push elements}
            \begin{tabular}{|c|c|}
                  \hline
                  Collection type & Time      \\
                  \hline
                  Vec             & 104.36 µs \\
                  Std LL          & 1.3043 ms \\
                  LL              & 1.2673 ms \\
                  Immutable LL    & 1.4542 ms \\
                  Thread Safe LL  & 3.6213 ms \\
                  DLL             & 2.5458 ms \\
                  \hline
            \end{tabular}
      \end{center}
\end{table}

In this table we can see how long it took to pop 100000 integers from a Vec,\\
std::collections::LinkedList, Linked List, Immutable Linked List, Thread Safe
Linked List and Doubly Linked List.
\begin{table}[!h]
      \begin{center}
            \caption{Pop elements}
            \begin{tabular}{|c|c|}
                  \hline
                  Collection type & Time      \\
                  \hline
                  Vec             & 17.773 µs \\
                  Std LL          & 693.14 µs \\
                  LL              & 663.92 µs \\
                  Immutable LL    & 794.60 µs \\
                  Thread Safe LL  & 2.7304 ms \\
                  DLL             & 1.5427 ms \\
                  \hline
            \end{tabular}
      \end{center}
\end{table}

In this table we can see how long it took to get element in the middle of a Vec,
Linked List, Immutable Linked List, Thread Safe Linked List with 100000 elements.
Std::collections::LinkedList and our Doubly Linked List don't support this operation
\begin{table}[!h]
      \begin{center}
            \caption{Pop elements}
            \begin{tabular}{|c|c|}
                  \hline
                  Collection type & Time      \\
                  \hline
                  Vec             & 190.43 ps \\
                  LL              & 102.73 µs \\
                  Immutable LL    & 156.59 µs \\
                  Thread Safe LL  & 149.29 µs \\
                  \hline
            \end{tabular}
      \end{center}
\end{table}

\pagebreak
\subsection*{When Linked Lists don't suck}

In this table we can see how long it took to push 100 $[u64; 10000]$ into a Vec,\\
std::collections::LinkedList, Linked List, Immutable Linked List, Thread Safe
Linked List and Doubly Linked List.
\footnote{All of these times also include the time of cloning the whole array, because
      when we wanted to create all to be inserted arrays, we got stack overflow error}
\begin{table}[!h]
      \begin{center}
            \caption{Pop elements}
            \begin{tabular}{|c|c|}
                  \hline
                  Collection type & Time      \\
                  \hline
                  Vec             & 537.76 µs \\
                  Std LL          & 464.61 µs \\
                  LL              & 474.34 µs \\
                  Immutable LL    & 627.81 µs \\
                  Thread Safe LL  & 782.17 µs \\
                  DLL             & 1.1396 ms \\
                  \hline
            \end{tabular}
      \end{center}
\end{table}

In this table we can see how long it took to pop 100 $[u64; 10000]$ from a Vec,\\
std::collections::LinkedList, Linked List, Immutable Linked List, Thread Safe
Linked List and Doubly Linked List.
\begin{table}[!h]
      \begin{center}
            \caption{Pop elements}
            \begin{tabular}{|c|c|}
                  \hline
                  Collection type & Time      \\
                  \hline
                  Vec             & 656.30 µs \\
                  Std LL          & 469.21 µs \\
                  LL              & 455.28 µs \\
                  Immutable LL    & 2.7239 µs \\
                  Thread Safe LL  & 320.21 µs \\
                  DLL             & 917.19 µs \\
                  \hline
            \end{tabular}
      \end{center}
\end{table}

In this table we can see how long it took to get element in the middle of a Vec,\\
Linked List, Immutable Linked List, Thread Safe Linked List with 100
$[u64; 10000]$ elements. Std::collections::LinkedList and our Doubly Linked List
don't support this operation
\begin{table}[!h]
      \begin{center}
            \caption{Pop elements}
            \begin{tabular}{|c|c|}
                  \hline
                  Collection type & Time      \\
                  \hline
                  Vec             & 184.92 ps \\
                  LL              & 27.667 ns \\
                  Immutable LL    & 70.145 ns \\
                  Thread Safe LL  & 70.208 ns \\
                  \hline
            \end{tabular}
      \end{center}
\end{table}

\section{Conclusion}

When we were choosing what to do for our project, we laughed at the suggestion
of implementing a linked list. After a call with our tutors, in which they
explained to us why linked lists in Rust aren't trivial we choose this topic
thinking that they are just fooling us. Well after couple of weekends fighting
with the borrow checker, we now understand why linked lists and Rusts safety
don't mix well together.

As we learned during the implementation, linked lists aren't that useful (except in some
rare scenarios). And we showed benchmarks proving that.

Thanks to this we learned, in more depth, how we can differently share references
in more complex scenarios. Such as in doubly linked list, which contains many
small cycles. Also we learned how to use criterion with more complicated benchmarks,
and how to tune the benchmark environment for more accurate results.

\bibliography{sources}
\bibliographystyle{ieeetr}

\end{document}
